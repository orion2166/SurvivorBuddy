%% Generated by Sphinx.
\def\sphinxdocclass{report}
\documentclass[letterpaper,10pt,english,openany,oneside]{sphinxmanual}
\ifdefined\pdfpxdimen
   \let\sphinxpxdimen\pdfpxdimen\else\newdimen\sphinxpxdimen
\fi \sphinxpxdimen=.75bp\relax

\PassOptionsToPackage{warn}{textcomp}
\usepackage[utf8]{inputenc}
\ifdefined\DeclareUnicodeCharacter
% support both utf8 and utf8x syntaxes
  \ifdefined\DeclareUnicodeCharacterAsOptional
    \def\sphinxDUC#1{\DeclareUnicodeCharacter{"#1}}
  \else
    \let\sphinxDUC\DeclareUnicodeCharacter
  \fi
  \sphinxDUC{00A0}{\nobreakspace}
  \sphinxDUC{2500}{\sphinxunichar{2500}}
  \sphinxDUC{2502}{\sphinxunichar{2502}}
  \sphinxDUC{2514}{\sphinxunichar{2514}}
  \sphinxDUC{251C}{\sphinxunichar{251C}}
  \sphinxDUC{2572}{\textbackslash}
\fi
\usepackage{cmap}
\usepackage[T1]{fontenc}
\usepackage{amsmath,amssymb,amstext}
\usepackage{babel}



\usepackage{times}
\expandafter\ifx\csname T@LGR\endcsname\relax
\else
% LGR was declared as font encoding
  \substitutefont{LGR}{\rmdefault}{cmr}
  \substitutefont{LGR}{\sfdefault}{cmss}
  \substitutefont{LGR}{\ttdefault}{cmtt}
\fi
\expandafter\ifx\csname T@X2\endcsname\relax
  \expandafter\ifx\csname T@T2A\endcsname\relax
  \else
  % T2A was declared as font encoding
    \substitutefont{T2A}{\rmdefault}{cmr}
    \substitutefont{T2A}{\sfdefault}{cmss}
    \substitutefont{T2A}{\ttdefault}{cmtt}
  \fi
\else
% X2 was declared as font encoding
  \substitutefont{X2}{\rmdefault}{cmr}
  \substitutefont{X2}{\sfdefault}{cmss}
  \substitutefont{X2}{\ttdefault}{cmtt}
\fi


\usepackage[Bjarne]{fncychap}
\usepackage{sphinx}

\fvset{fontsize=\small}
\usepackage{geometry}


% Include hyperref last.
\usepackage{hyperref}
% Fix anchor placement for figures with captions.
\usepackage{hypcap}% it must be loaded after hyperref.
% Set up styles of URL: it should be placed after hyperref.
\urlstyle{same}
\addto\captionsenglish{\renewcommand{\contentsname}{Contents:}}

\usepackage{sphinxmessages}




\title{Survivor Buddy 3.0}
\date{Apr 28, 2020}
\release{0.1.0}
\author{Ben Shiller, Joseph Duran, Philip Rettenmaier, Yara Mohamed, Kimberly Ramos}
\newcommand{\sphinxlogo}{\vbox{}}
\renewcommand{\releasename}{Release}
\makeindex
\begin{document}

\pagestyle{empty}
\sphinxmaketitle
\pagestyle{plain}
\sphinxtableofcontents
\pagestyle{normal}
\phantomsection\label{\detokenize{index::doc}}



\chapter{Application}
\label{\detokenize{src/application:module-Application}}\label{\detokenize{src/application:application}}\label{\detokenize{src/application::doc}}\index{module@\spxentry{module}!Application@\spxentry{Application}}\index{Application@\spxentry{Application}!module@\spxentry{module}}\index{Application (class in Application)@\spxentry{Application}\spxextra{class in Application}}

\begin{fulllineitems}
\phantomsection\label{\detokenize{src/application:Application.Application}}\pysiglinewithargsret{\sphinxbfcode{\sphinxupquote{class }}\sphinxcode{\sphinxupquote{Application.}}\sphinxbfcode{\sphinxupquote{Application}}}{\emph{\DUrole{n}{master}}, \emph{\DUrole{o}{**}\DUrole{n}{kwargs}}}{}
The main GUI class
\index{\_\_init\_\_() (Application.Application method)@\spxentry{\_\_init\_\_()}\spxextra{Application.Application method}}

\begin{fulllineitems}
\phantomsection\label{\detokenize{src/application:Application.Application.__init__}}\pysiglinewithargsret{\sphinxbfcode{\sphinxupquote{\_\_init\_\_}}}{\emph{\DUrole{n}{master}}, \emph{\DUrole{o}{**}\DUrole{n}{kwargs}}}{}
The constructor for the Application class
\begin{quote}\begin{description}
\item[{Parameters}] \leavevmode
\sphinxstyleliteralstrong{\sphinxupquote{master}} \textendash{} the Tk parent widget

\end{description}\end{quote}

\end{fulllineitems}

\index{create\_widgets() (Application.Application method)@\spxentry{create\_widgets()}\spxextra{Application.Application method}}

\begin{fulllineitems}
\phantomsection\label{\detokenize{src/application:Application.Application.create_widgets}}\pysiglinewithargsret{\sphinxbfcode{\sphinxupquote{create\_widgets}}}{}{}
Creates the widgets seen in the GUI

\end{fulllineitems}

\index{close\_app() (Application.Application method)@\spxentry{close\_app()}\spxextra{Application.Application method}}

\begin{fulllineitems}
\phantomsection\label{\detokenize{src/application:Application.Application.close_app}}\pysiglinewithargsret{\sphinxbfcode{\sphinxupquote{close\_app}}}{}{}
Closes the GUI application

\end{fulllineitems}

\index{create\_menu() (Application.Application method)@\spxentry{create\_menu()}\spxextra{Application.Application method}}

\begin{fulllineitems}
\phantomsection\label{\detokenize{src/application:Application.Application.create_menu}}\pysiglinewithargsret{\sphinxbfcode{\sphinxupquote{create\_menu}}}{\emph{\DUrole{n}{root\_menu}}}{}
Creates the main GUI menu
\begin{quote}\begin{description}
\item[{Parameters}] \leavevmode
\sphinxstyleliteralstrong{\sphinxupquote{root\_menu}} \textendash{} The root menu (self.menu\_bar) that is instantiated in create\_widgets()

\end{description}\end{quote}

\end{fulllineitems}

\index{refresh\_devices() (Application.Application method)@\spxentry{refresh\_devices()}\spxextra{Application.Application method}}

\begin{fulllineitems}
\phantomsection\label{\detokenize{src/application:Application.Application.refresh_devices}}\pysiglinewithargsret{\sphinxbfcode{\sphinxupquote{refresh\_devices}}}{}{}
Refreshes the Devices menu

\end{fulllineitems}

\index{connect() (Application.Application method)@\spxentry{connect()}\spxextra{Application.Application method}}

\begin{fulllineitems}
\phantomsection\label{\detokenize{src/application:Application.Application.connect}}\pysiglinewithargsret{\sphinxbfcode{\sphinxupquote{connect}}}{\emph{\DUrole{n}{dev}}}{}
Connects to the given device
\begin{quote}\begin{description}
\item[{Parameters}] \leavevmode
\sphinxstyleliteralstrong{\sphinxupquote{dev}} \textendash{} The serial device to connect to

\end{description}\end{quote}

\end{fulllineitems}

\index{close() (Application.Application method)@\spxentry{close()}\spxextra{Application.Application method}}

\begin{fulllineitems}
\phantomsection\label{\detokenize{src/application:Application.Application.close}}\pysiglinewithargsret{\sphinxbfcode{\sphinxupquote{close}}}{}{}
Closes the active serial connection

\end{fulllineitems}

\index{hello() (Application.Application method)@\spxentry{hello()}\spxextra{Application.Application method}}

\begin{fulllineitems}
\phantomsection\label{\detokenize{src/application:Application.Application.hello}}\pysiglinewithargsret{\sphinxbfcode{\sphinxupquote{hello}}}{}{}
A test function

Simply prints “Hello from Menu” to the console and the NotificationsFrame

\end{fulllineitems}


\end{fulllineitems}



\chapter{ControlButtons}
\label{\detokenize{src/controlbuttons:module-ControlButtons}}\label{\detokenize{src/controlbuttons:controlbuttons}}\label{\detokenize{src/controlbuttons::doc}}\index{module@\spxentry{module}!ControlButtons@\spxentry{ControlButtons}}\index{ControlButtons@\spxentry{ControlButtons}!module@\spxentry{module}}\index{ControlButtons (class in ControlButtons)@\spxentry{ControlButtons}\spxextra{class in ControlButtons}}

\begin{fulllineitems}
\phantomsection\label{\detokenize{src/controlbuttons:ControlButtons.ControlButtons}}\pysiglinewithargsret{\sphinxbfcode{\sphinxupquote{class }}\sphinxcode{\sphinxupquote{ControlButtons.}}\sphinxbfcode{\sphinxupquote{ControlButtons}}}{\emph{\DUrole{n}{master}}, \emph{\DUrole{n}{arm\_controller}}, \emph{\DUrole{n}{notifications}}, \emph{\DUrole{o}{**}\DUrole{n}{kwargs}}}{}
Buttons to control the Survivor Buddy 3.0 arm
\index{\_\_init\_\_() (ControlButtons.ControlButtons method)@\spxentry{\_\_init\_\_()}\spxextra{ControlButtons.ControlButtons method}}

\begin{fulllineitems}
\phantomsection\label{\detokenize{src/controlbuttons:ControlButtons.ControlButtons.__init__}}\pysiglinewithargsret{\sphinxbfcode{\sphinxupquote{\_\_init\_\_}}}{\emph{\DUrole{n}{master}}, \emph{\DUrole{n}{arm\_controller}}, \emph{\DUrole{n}{notifications}}, \emph{\DUrole{o}{**}\DUrole{n}{kwargs}}}{}
The constructor for ControlButtons
\begin{quote}\begin{description}
\item[{Parameters}] \leavevmode\begin{itemize}
\item {} 
\sphinxstyleliteralstrong{\sphinxupquote{master}} \textendash{} The Tk parent widget

\item {} 
\sphinxstyleliteralstrong{\sphinxupquote{arm\_controller}} \textendash{} The SerialArmController being used

\item {} 
\sphinxstyleliteralstrong{\sphinxupquote{notifications}} \textendash{} The NotificationsFrame being used

\end{itemize}

\end{description}\end{quote}

\end{fulllineitems}

\index{create\_buttons() (ControlButtons.ControlButtons method)@\spxentry{create\_buttons()}\spxextra{ControlButtons.ControlButtons method}}

\begin{fulllineitems}
\phantomsection\label{\detokenize{src/controlbuttons:ControlButtons.ControlButtons.create_buttons}}\pysiglinewithargsret{\sphinxbfcode{\sphinxupquote{create\_buttons}}}{}{}
Creates the control buttons displayed in the GUI

\end{fulllineitems}

\index{open\_arm() (ControlButtons.ControlButtons method)@\spxentry{open\_arm()}\spxextra{ControlButtons.ControlButtons method}}

\begin{fulllineitems}
\phantomsection\label{\detokenize{src/controlbuttons:ControlButtons.ControlButtons.open_arm}}\pysiglinewithargsret{\sphinxbfcode{\sphinxupquote{open\_arm}}}{}{}
Opens the arm using SerialArmController

\end{fulllineitems}

\index{close\_arm() (ControlButtons.ControlButtons method)@\spxentry{close\_arm()}\spxextra{ControlButtons.ControlButtons method}}

\begin{fulllineitems}
\phantomsection\label{\detokenize{src/controlbuttons:ControlButtons.ControlButtons.close_arm}}\pysiglinewithargsret{\sphinxbfcode{\sphinxupquote{close\_arm}}}{}{}
Closes the arm using SerialArmController

\end{fulllineitems}

\index{portrait() (ControlButtons.ControlButtons method)@\spxentry{portrait()}\spxextra{ControlButtons.ControlButtons method}}

\begin{fulllineitems}
\phantomsection\label{\detokenize{src/controlbuttons:ControlButtons.ControlButtons.portrait}}\pysiglinewithargsret{\sphinxbfcode{\sphinxupquote{portrait}}}{}{}
Changes the arm to portrait mode using SerialArmController

\end{fulllineitems}

\index{landscape() (ControlButtons.ControlButtons method)@\spxentry{landscape()}\spxextra{ControlButtons.ControlButtons method}}

\begin{fulllineitems}
\phantomsection\label{\detokenize{src/controlbuttons:ControlButtons.ControlButtons.landscape}}\pysiglinewithargsret{\sphinxbfcode{\sphinxupquote{landscape}}}{}{}
Changes the arm to landscape mode using SerialArmController

\end{fulllineitems}

\index{tilt() (ControlButtons.ControlButtons method)@\spxentry{tilt()}\spxextra{ControlButtons.ControlButtons method}}

\begin{fulllineitems}
\phantomsection\label{\detokenize{src/controlbuttons:ControlButtons.ControlButtons.tilt}}\pysiglinewithargsret{\sphinxbfcode{\sphinxupquote{tilt}}}{}{}
Tilts the arm using SerialArmController

\end{fulllineitems}

\index{nod() (ControlButtons.ControlButtons method)@\spxentry{nod()}\spxextra{ControlButtons.ControlButtons method}}

\begin{fulllineitems}
\phantomsection\label{\detokenize{src/controlbuttons:ControlButtons.ControlButtons.nod}}\pysiglinewithargsret{\sphinxbfcode{\sphinxupquote{nod}}}{}{}
Nods the arm using SerialArmController

\end{fulllineitems}

\index{shake() (ControlButtons.ControlButtons method)@\spxentry{shake()}\spxextra{ControlButtons.ControlButtons method}}

\begin{fulllineitems}
\phantomsection\label{\detokenize{src/controlbuttons:ControlButtons.ControlButtons.shake}}\pysiglinewithargsret{\sphinxbfcode{\sphinxupquote{shake}}}{}{}
Shakes the arm using SerialArmController

\end{fulllineitems}

\index{shutdown() (ControlButtons.ControlButtons method)@\spxentry{shutdown()}\spxextra{ControlButtons.ControlButtons method}}

\begin{fulllineitems}
\phantomsection\label{\detokenize{src/controlbuttons:ControlButtons.ControlButtons.shutdown}}\pysiglinewithargsret{\sphinxbfcode{\sphinxupquote{shutdown}}}{}{}
Shuts down the arm using SerialArmController

\end{fulllineitems}


\end{fulllineitems}



\chapter{NotificationsFrame}
\label{\detokenize{src/notificationsframe:module-NotificationsFrame}}\label{\detokenize{src/notificationsframe:notificationsframe}}\label{\detokenize{src/notificationsframe::doc}}\index{module@\spxentry{module}!NotificationsFrame@\spxentry{NotificationsFrame}}\index{NotificationsFrame@\spxentry{NotificationsFrame}!module@\spxentry{module}}\index{NotificationFrame (class in NotificationsFrame)@\spxentry{NotificationFrame}\spxextra{class in NotificationsFrame}}

\begin{fulllineitems}
\phantomsection\label{\detokenize{src/notificationsframe:NotificationsFrame.NotificationFrame}}\pysiglinewithargsret{\sphinxbfcode{\sphinxupquote{class }}\sphinxcode{\sphinxupquote{NotificationsFrame.}}\sphinxbfcode{\sphinxupquote{NotificationFrame}}}{\emph{\DUrole{n}{master}}, \emph{\DUrole{n}{\_logFile}}, \emph{\DUrole{o}{**}\DUrole{n}{kwargs}}}{}
Box to display notification in the GUI
\index{\_\_init\_\_() (NotificationsFrame.NotificationFrame method)@\spxentry{\_\_init\_\_()}\spxextra{NotificationsFrame.NotificationFrame method}}

\begin{fulllineitems}
\phantomsection\label{\detokenize{src/notificationsframe:NotificationsFrame.NotificationFrame.__init__}}\pysiglinewithargsret{\sphinxbfcode{\sphinxupquote{\_\_init\_\_}}}{\emph{\DUrole{n}{master}}, \emph{\DUrole{n}{\_logFile}}, \emph{\DUrole{o}{**}\DUrole{n}{kwargs}}}{}
The constructor for NotificationsFrame
\begin{quote}\begin{description}
\item[{Parameters}] \leavevmode\begin{itemize}
\item {} 
\sphinxstyleliteralstrong{\sphinxupquote{master}} \textendash{} The Tk parent widget

\item {} 
\sphinxstyleliteralstrong{\sphinxupquote{\_logFile}} \textendash{} The file handle for the output log file

\end{itemize}

\end{description}\end{quote}

\end{fulllineitems}

\index{append\_line() (NotificationsFrame.NotificationFrame method)@\spxentry{append\_line()}\spxextra{NotificationsFrame.NotificationFrame method}}

\begin{fulllineitems}
\phantomsection\label{\detokenize{src/notificationsframe:NotificationsFrame.NotificationFrame.append_line}}\pysiglinewithargsret{\sphinxbfcode{\sphinxupquote{append\_line}}}{\emph{\DUrole{n}{line}}}{}
Prints a line to the notification box and 
a timestamped line to the log file

\end{fulllineitems}


\end{fulllineitems}



\chapter{PositionFrame}
\label{\detokenize{src/positionframe:module-PositionFrame}}\label{\detokenize{src/positionframe:positionframe}}\label{\detokenize{src/positionframe::doc}}\index{module@\spxentry{module}!PositionFrame@\spxentry{PositionFrame}}\index{PositionFrame@\spxentry{PositionFrame}!module@\spxentry{module}}\index{PositionUpdater (class in PositionFrame)@\spxentry{PositionUpdater}\spxextra{class in PositionFrame}}

\begin{fulllineitems}
\phantomsection\label{\detokenize{src/positionframe:PositionFrame.PositionUpdater}}\pysiglinewithargsret{\sphinxbfcode{\sphinxupquote{class }}\sphinxcode{\sphinxupquote{PositionFrame.}}\sphinxbfcode{\sphinxupquote{PositionUpdater}}}{\emph{\DUrole{n}{dev}}, \emph{\DUrole{n}{\_pitch\_control}}, \emph{\DUrole{n}{\_yaw\_control}}, \emph{\DUrole{n}{\_roll\_control}}, \emph{\DUrole{n}{\_yaw\_queue}}, \emph{\DUrole{n}{\_pitch\_queue}}, \emph{\DUrole{n}{\_roll\_queue}}, \emph{\DUrole{n}{\_notifications}}, \emph{\DUrole{o}{**}\DUrole{n}{kwargs}}}{}
Updates UI elements based on arm position
\index{\_\_init\_\_() (PositionFrame.PositionUpdater method)@\spxentry{\_\_init\_\_()}\spxextra{PositionFrame.PositionUpdater method}}

\begin{fulllineitems}
\phantomsection\label{\detokenize{src/positionframe:PositionFrame.PositionUpdater.__init__}}\pysiglinewithargsret{\sphinxbfcode{\sphinxupquote{\_\_init\_\_}}}{\emph{\DUrole{n}{dev}}, \emph{\DUrole{n}{\_pitch\_control}}, \emph{\DUrole{n}{\_yaw\_control}}, \emph{\DUrole{n}{\_roll\_control}}, \emph{\DUrole{n}{\_yaw\_queue}}, \emph{\DUrole{n}{\_pitch\_queue}}, \emph{\DUrole{n}{\_roll\_queue}}, \emph{\DUrole{n}{\_notifications}}, \emph{\DUrole{o}{**}\DUrole{n}{kwargs}}}{}
Constructor for PositionUpdater
\begin{quote}\begin{description}
\item[{Parameters}] \leavevmode\begin{itemize}
\item {} 
\sphinxstyleliteralstrong{\sphinxupquote{dev}} \textendash{} The SerialArmController

\item {} 
\sphinxstyleliteralstrong{\sphinxupquote{\_pitch\_control}} \textendash{} The pitch LabelScaleSpinbox

\item {} 
\sphinxstyleliteralstrong{\sphinxupquote{\_yaw\_control}} \textendash{} The yaw LabelScaleSpinbox

\item {} 
\sphinxstyleliteralstrong{\sphinxupquote{\_roll\_control}} \textendash{} The roll LabelScaleSpinbox

\item {} 
\sphinxstyleliteralstrong{\sphinxupquote{\_yaw\_queue}} \textendash{} The yaw queue

\item {} 
\sphinxstyleliteralstrong{\sphinxupquote{\_pitch\_queue}} \textendash{} The pitch queue

\item {} 
\sphinxstyleliteralstrong{\sphinxupquote{\_roll\_queue}} \textendash{} The roll queue

\item {} 
\sphinxstyleliteralstrong{\sphinxupquote{\_notifications}} \textendash{} The NotificationsFrame

\end{itemize}

\end{description}\end{quote}

\end{fulllineitems}

\index{run() (PositionFrame.PositionUpdater method)@\spxentry{run()}\spxextra{PositionFrame.PositionUpdater method}}

\begin{fulllineitems}
\phantomsection\label{\detokenize{src/positionframe:PositionFrame.PositionUpdater.run}}\pysiglinewithargsret{\sphinxbfcode{\sphinxupquote{run}}}{}{}
Continually checks for changes to arm position, then updates the UI based on these changes, checks for changes every 0.1 seconds.
Runs as a thread separately from rest of UI, uses queue to update render and directly updates sliders and spinboxes.

\end{fulllineitems}


\end{fulllineitems}

\index{LabelScaleSpinbox (class in PositionFrame)@\spxentry{LabelScaleSpinbox}\spxextra{class in PositionFrame}}

\begin{fulllineitems}
\phantomsection\label{\detokenize{src/positionframe:PositionFrame.LabelScaleSpinbox}}\pysiglinewithargsret{\sphinxbfcode{\sphinxupquote{class }}\sphinxcode{\sphinxupquote{PositionFrame.}}\sphinxbfcode{\sphinxupquote{LabelScaleSpinbox}}}{\emph{\DUrole{n}{master}}, \emph{\DUrole{n}{text}\DUrole{o}{=}\DUrole{default_value}{\textquotesingle{}\textquotesingle{}}}, \emph{\DUrole{n}{from\_}\DUrole{o}{=}\DUrole{default_value}{0}}, \emph{\DUrole{n}{to}\DUrole{o}{=}\DUrole{default_value}{10}}, \emph{\DUrole{n}{axis}\DUrole{o}{=}\DUrole{default_value}{0}}, \emph{\DUrole{n}{dev}\DUrole{o}{=}\DUrole{default_value}{None}}, \emph{\DUrole{o}{**}\DUrole{n}{kwargs}}}{}
A custom class to combine Tk Scale and Spinbox and keep them in sync
\index{\_\_init\_\_() (PositionFrame.LabelScaleSpinbox method)@\spxentry{\_\_init\_\_()}\spxextra{PositionFrame.LabelScaleSpinbox method}}

\begin{fulllineitems}
\phantomsection\label{\detokenize{src/positionframe:PositionFrame.LabelScaleSpinbox.__init__}}\pysiglinewithargsret{\sphinxbfcode{\sphinxupquote{\_\_init\_\_}}}{\emph{\DUrole{n}{master}}, \emph{\DUrole{n}{text}\DUrole{o}{=}\DUrole{default_value}{\textquotesingle{}\textquotesingle{}}}, \emph{\DUrole{n}{from\_}\DUrole{o}{=}\DUrole{default_value}{0}}, \emph{\DUrole{n}{to}\DUrole{o}{=}\DUrole{default_value}{10}}, \emph{\DUrole{n}{axis}\DUrole{o}{=}\DUrole{default_value}{0}}, \emph{\DUrole{n}{dev}\DUrole{o}{=}\DUrole{default_value}{None}}, \emph{\DUrole{o}{**}\DUrole{n}{kwargs}}}{}
Constructor for LabelScaleSpinbox
\begin{quote}\begin{description}
\item[{Parameters}] \leavevmode\begin{itemize}
\item {} 
\sphinxstyleliteralstrong{\sphinxupquote{master}} \textendash{} The Tk parent widget

\item {} 
\sphinxstyleliteralstrong{\sphinxupquote{text}} \textendash{} The text to display next to the control

\item {} 
\sphinxstyleliteralstrong{\sphinxupquote{from}} \textendash{} The minimum valid value

\item {} 
\sphinxstyleliteralstrong{\sphinxupquote{to}} \textendash{} The maximum valid value

\item {} 
\sphinxstyleliteralstrong{\sphinxupquote{axis}} \textendash{} The axis that this LabelScaleSpinbox controls

\item {} 
\sphinxstyleliteralstrong{\sphinxupquote{dev}} \textendash{} The SerialArmController

\end{itemize}

\end{description}\end{quote}

\end{fulllineitems}

\index{sliderUpdate() (PositionFrame.LabelScaleSpinbox method)@\spxentry{sliderUpdate()}\spxextra{PositionFrame.LabelScaleSpinbox method}}

\begin{fulllineitems}
\phantomsection\label{\detokenize{src/positionframe:PositionFrame.LabelScaleSpinbox.sliderUpdate}}\pysiglinewithargsret{\sphinxbfcode{\sphinxupquote{sliderUpdate}}}{\emph{\DUrole{n}{val}}}{}
Sends command to arm based on slider value, sets spinbox based on slider
\begin{quote}\begin{description}
\item[{Parameters}] \leavevmode
\sphinxstyleliteralstrong{\sphinxupquote{val}} \textendash{} The value passed to this function when the slider is released

\end{description}\end{quote}

\end{fulllineitems}

\index{validate\_spinbox() (PositionFrame.LabelScaleSpinbox method)@\spxentry{validate\_spinbox()}\spxextra{PositionFrame.LabelScaleSpinbox method}}

\begin{fulllineitems}
\phantomsection\label{\detokenize{src/positionframe:PositionFrame.LabelScaleSpinbox.validate_spinbox}}\pysiglinewithargsret{\sphinxbfcode{\sphinxupquote{validate\_spinbox}}}{\emph{\DUrole{n}{val}}}{}
Check that spinbox and slider are within valid range of values
\begin{quote}\begin{description}
\item[{Parameters}] \leavevmode
\sphinxstyleliteralstrong{\sphinxupquote{val}} \textendash{} The value from the spinbox

\end{description}\end{quote}

\end{fulllineitems}

\index{invalid\_spinbox() (PositionFrame.LabelScaleSpinbox method)@\spxentry{invalid\_spinbox()}\spxextra{PositionFrame.LabelScaleSpinbox method}}

\begin{fulllineitems}
\phantomsection\label{\detokenize{src/positionframe:PositionFrame.LabelScaleSpinbox.invalid_spinbox}}\pysiglinewithargsret{\sphinxbfcode{\sphinxupquote{invalid\_spinbox}}}{}{}
Function that runs when the spinbox has an invalid value

\end{fulllineitems}

\index{set\_slider() (PositionFrame.LabelScaleSpinbox method)@\spxentry{set\_slider()}\spxextra{PositionFrame.LabelScaleSpinbox method}}

\begin{fulllineitems}
\phantomsection\label{\detokenize{src/positionframe:PositionFrame.LabelScaleSpinbox.set_slider}}\pysiglinewithargsret{\sphinxbfcode{\sphinxupquote{set\_slider}}}{}{}
Set slider position based on spinbox value, send command to arm

\end{fulllineitems}

\index{send\_command() (PositionFrame.LabelScaleSpinbox method)@\spxentry{send\_command()}\spxextra{PositionFrame.LabelScaleSpinbox method}}

\begin{fulllineitems}
\phantomsection\label{\detokenize{src/positionframe:PositionFrame.LabelScaleSpinbox.send_command}}\pysiglinewithargsret{\sphinxbfcode{\sphinxupquote{send\_command}}}{}{}
Sends a new position to the arm based on changed axis

\end{fulllineitems}


\end{fulllineitems}

\index{RenderDiagram (class in PositionFrame)@\spxentry{RenderDiagram}\spxextra{class in PositionFrame}}

\begin{fulllineitems}
\phantomsection\label{\detokenize{src/positionframe:PositionFrame.RenderDiagram}}\pysiglinewithargsret{\sphinxbfcode{\sphinxupquote{class }}\sphinxcode{\sphinxupquote{PositionFrame.}}\sphinxbfcode{\sphinxupquote{RenderDiagram}}}{\emph{\DUrole{n}{master}}, \emph{\DUrole{n}{dev}\DUrole{o}{=}\DUrole{default_value}{None}}, \emph{\DUrole{o}{**}\DUrole{n}{kwargs}}}{}
Displays a basic render of arm, helps to show position when arm can not be seen by user
\index{\_\_init\_\_() (PositionFrame.RenderDiagram method)@\spxentry{\_\_init\_\_()}\spxextra{PositionFrame.RenderDiagram method}}

\begin{fulllineitems}
\phantomsection\label{\detokenize{src/positionframe:PositionFrame.RenderDiagram.__init__}}\pysiglinewithargsret{\sphinxbfcode{\sphinxupquote{\_\_init\_\_}}}{\emph{\DUrole{n}{master}}, \emph{\DUrole{n}{dev}\DUrole{o}{=}\DUrole{default_value}{None}}, \emph{\DUrole{o}{**}\DUrole{n}{kwargs}}}{}
Constructor for RenderDiagram
\begin{quote}\begin{description}
\item[{Parameters}] \leavevmode\begin{itemize}
\item {} 
\sphinxstyleliteralstrong{\sphinxupquote{master}} \textendash{} The Tk parent widget

\item {} 
\sphinxstyleliteralstrong{\sphinxupquote{dev}} \textendash{} The SerialArmController

\end{itemize}

\end{description}\end{quote}

\end{fulllineitems}

\index{draw\_axes() (PositionFrame.RenderDiagram method)@\spxentry{draw\_axes()}\spxextra{PositionFrame.RenderDiagram method}}

\begin{fulllineitems}
\phantomsection\label{\detokenize{src/positionframe:PositionFrame.RenderDiagram.draw_axes}}\pysiglinewithargsret{\sphinxbfcode{\sphinxupquote{draw\_axes}}}{}{}
Clear units from axes, display arm base, set axis limits

\end{fulllineitems}

\index{update\_render() (PositionFrame.RenderDiagram method)@\spxentry{update\_render()}\spxextra{PositionFrame.RenderDiagram method}}

\begin{fulllineitems}
\phantomsection\label{\detokenize{src/positionframe:PositionFrame.RenderDiagram.update_render}}\pysiglinewithargsret{\sphinxbfcode{\sphinxupquote{update\_render}}}{\emph{\DUrole{n}{master}}, \emph{\DUrole{n}{new\_yaw}}, \emph{\DUrole{n}{new\_pitch}}, \emph{\DUrole{n}{new\_roll}}}{}
Update display of render based on new arm position
\begin{quote}\begin{description}
\item[{Parameters}] \leavevmode\begin{itemize}
\item {} 
\sphinxstyleliteralstrong{\sphinxupquote{master}} \textendash{} The Tk parent widget

\item {} 
\sphinxstyleliteralstrong{\sphinxupquote{new\_yaw}} \textendash{} The new yaw value

\item {} 
\sphinxstyleliteralstrong{\sphinxupquote{new\_pitch}} \textendash{} The new pitch value

\item {} 
\sphinxstyleliteralstrong{\sphinxupquote{new\_roll}} \textendash{} The new roll value

\end{itemize}

\end{description}\end{quote}

\end{fulllineitems}


\end{fulllineitems}

\index{PositionFrame (class in PositionFrame)@\spxentry{PositionFrame}\spxextra{class in PositionFrame}}

\begin{fulllineitems}
\phantomsection\label{\detokenize{src/positionframe:PositionFrame.PositionFrame}}\pysiglinewithargsret{\sphinxbfcode{\sphinxupquote{class }}\sphinxcode{\sphinxupquote{PositionFrame.}}\sphinxbfcode{\sphinxupquote{PositionFrame}}}{\emph{\DUrole{n}{master}}, \emph{\DUrole{n}{arm\_controller}}, \emph{\DUrole{n}{\_logFile}}, \emph{\DUrole{o}{**}\DUrole{n}{kwargs}}}{}
Creates the Render and Control Sliders in the GUI
\index{\_\_init\_\_() (PositionFrame.PositionFrame method)@\spxentry{\_\_init\_\_()}\spxextra{PositionFrame.PositionFrame method}}

\begin{fulllineitems}
\phantomsection\label{\detokenize{src/positionframe:PositionFrame.PositionFrame.__init__}}\pysiglinewithargsret{\sphinxbfcode{\sphinxupquote{\_\_init\_\_}}}{\emph{\DUrole{n}{master}}, \emph{\DUrole{n}{arm\_controller}}, \emph{\DUrole{n}{\_logFile}}, \emph{\DUrole{o}{**}\DUrole{n}{kwargs}}}{}
Constructor for PositionFrame
\begin{quote}\begin{description}
\item[{Parameters}] \leavevmode\begin{itemize}
\item {} 
\sphinxstyleliteralstrong{\sphinxupquote{master}} \textendash{} The Tk parent widget

\item {} 
\sphinxstyleliteralstrong{\sphinxupquote{arm\_controller}} \textendash{} The SerialArmController

\item {} 
\sphinxstyleliteralstrong{\sphinxupquote{\_logFile}} \textendash{} The output log file handle

\end{itemize}

\end{description}\end{quote}

\end{fulllineitems}

\index{create\_render() (PositionFrame.PositionFrame method)@\spxentry{create\_render()}\spxextra{PositionFrame.PositionFrame method}}

\begin{fulllineitems}
\phantomsection\label{\detokenize{src/positionframe:PositionFrame.PositionFrame.create_render}}\pysiglinewithargsret{\sphinxbfcode{\sphinxupquote{create\_render}}}{\emph{\DUrole{n}{master}}}{}
Initializes render of arm
\begin{quote}\begin{description}
\item[{Parameters}] \leavevmode
\sphinxstyleliteralstrong{\sphinxupquote{master}} \textendash{} The Tk parent widget

\end{description}\end{quote}

\end{fulllineitems}

\index{create\_controls() (PositionFrame.PositionFrame method)@\spxentry{create\_controls()}\spxextra{PositionFrame.PositionFrame method}}

\begin{fulllineitems}
\phantomsection\label{\detokenize{src/positionframe:PositionFrame.PositionFrame.create_controls}}\pysiglinewithargsret{\sphinxbfcode{\sphinxupquote{create\_controls}}}{\emph{\DUrole{n}{master}}}{}
Creates LabelScaleSpinbox controls
\begin{quote}\begin{description}
\item[{Parameters}] \leavevmode
\sphinxstyleliteralstrong{\sphinxupquote{master}} \textendash{} The Tk parent widget

\end{description}\end{quote}

\end{fulllineitems}

\index{create\_updater() (PositionFrame.PositionFrame method)@\spxentry{create\_updater()}\spxextra{PositionFrame.PositionFrame method}}

\begin{fulllineitems}
\phantomsection\label{\detokenize{src/positionframe:PositionFrame.PositionFrame.create_updater}}\pysiglinewithargsret{\sphinxbfcode{\sphinxupquote{create\_updater}}}{}{}
Starts updater function to update GUI based on current position

\end{fulllineitems}

\index{process\_queue() (PositionFrame.PositionFrame method)@\spxentry{process\_queue()}\spxextra{PositionFrame.PositionFrame method}}

\begin{fulllineitems}
\phantomsection\label{\detokenize{src/positionframe:PositionFrame.PositionFrame.process_queue}}\pysiglinewithargsret{\sphinxbfcode{\sphinxupquote{process\_queue}}}{}{}
Processes queue of position updates

Uses this queue data to print new position to log file, and to update render position

\end{fulllineitems}


\end{fulllineitems}



\chapter{SerialArmController}
\label{\detokenize{src/serialarmcontroller:module-SerialArmController}}\label{\detokenize{src/serialarmcontroller:serialarmcontroller}}\label{\detokenize{src/serialarmcontroller::doc}}\index{module@\spxentry{module}!SerialArmController@\spxentry{SerialArmController}}\index{SerialArmController@\spxentry{SerialArmController}!module@\spxentry{module}}\index{Command (class in SerialArmController)@\spxentry{Command}\spxextra{class in SerialArmController}}

\begin{fulllineitems}
\phantomsection\label{\detokenize{src/serialarmcontroller:SerialArmController.Command}}\pysigline{\sphinxbfcode{\sphinxupquote{class }}\sphinxcode{\sphinxupquote{SerialArmController.}}\sphinxbfcode{\sphinxupquote{Command}}}
A class to keep track of command numbers
\index{PITCH (SerialArmController.Command attribute)@\spxentry{PITCH}\spxextra{SerialArmController.Command attribute}}

\begin{fulllineitems}
\phantomsection\label{\detokenize{src/serialarmcontroller:SerialArmController.Command.PITCH}}\pysigline{\sphinxbfcode{\sphinxupquote{PITCH}}\sphinxbfcode{\sphinxupquote{ = 0}}}
\end{fulllineitems}

\index{YAW (SerialArmController.Command attribute)@\spxentry{YAW}\spxextra{SerialArmController.Command attribute}}

\begin{fulllineitems}
\phantomsection\label{\detokenize{src/serialarmcontroller:SerialArmController.Command.YAW}}\pysigline{\sphinxbfcode{\sphinxupquote{YAW}}\sphinxbfcode{\sphinxupquote{ = 1}}}
\end{fulllineitems}

\index{ROLL (SerialArmController.Command attribute)@\spxentry{ROLL}\spxextra{SerialArmController.Command attribute}}

\begin{fulllineitems}
\phantomsection\label{\detokenize{src/serialarmcontroller:SerialArmController.Command.ROLL}}\pysigline{\sphinxbfcode{\sphinxupquote{ROLL}}\sphinxbfcode{\sphinxupquote{ = 2}}}
\end{fulllineitems}

\index{CLOSE (SerialArmController.Command attribute)@\spxentry{CLOSE}\spxextra{SerialArmController.Command attribute}}

\begin{fulllineitems}
\phantomsection\label{\detokenize{src/serialarmcontroller:SerialArmController.Command.CLOSE}}\pysigline{\sphinxbfcode{\sphinxupquote{CLOSE}}\sphinxbfcode{\sphinxupquote{ = 3}}}
\end{fulllineitems}

\index{OPEN (SerialArmController.Command attribute)@\spxentry{OPEN}\spxextra{SerialArmController.Command attribute}}

\begin{fulllineitems}
\phantomsection\label{\detokenize{src/serialarmcontroller:SerialArmController.Command.OPEN}}\pysigline{\sphinxbfcode{\sphinxupquote{OPEN}}\sphinxbfcode{\sphinxupquote{ = 4}}}
\end{fulllineitems}

\index{PORTRAIT (SerialArmController.Command attribute)@\spxentry{PORTRAIT}\spxextra{SerialArmController.Command attribute}}

\begin{fulllineitems}
\phantomsection\label{\detokenize{src/serialarmcontroller:SerialArmController.Command.PORTRAIT}}\pysigline{\sphinxbfcode{\sphinxupquote{PORTRAIT}}\sphinxbfcode{\sphinxupquote{ = 5}}}
\end{fulllineitems}

\index{LANDSCAPE (SerialArmController.Command attribute)@\spxentry{LANDSCAPE}\spxextra{SerialArmController.Command attribute}}

\begin{fulllineitems}
\phantomsection\label{\detokenize{src/serialarmcontroller:SerialArmController.Command.LANDSCAPE}}\pysigline{\sphinxbfcode{\sphinxupquote{LANDSCAPE}}\sphinxbfcode{\sphinxupquote{ = 6}}}
\end{fulllineitems}

\index{NOD (SerialArmController.Command attribute)@\spxentry{NOD}\spxextra{SerialArmController.Command attribute}}

\begin{fulllineitems}
\phantomsection\label{\detokenize{src/serialarmcontroller:SerialArmController.Command.NOD}}\pysigline{\sphinxbfcode{\sphinxupquote{NOD}}\sphinxbfcode{\sphinxupquote{ = 7}}}
\end{fulllineitems}

\index{SHAKE (SerialArmController.Command attribute)@\spxentry{SHAKE}\spxextra{SerialArmController.Command attribute}}

\begin{fulllineitems}
\phantomsection\label{\detokenize{src/serialarmcontroller:SerialArmController.Command.SHAKE}}\pysigline{\sphinxbfcode{\sphinxupquote{SHAKE}}\sphinxbfcode{\sphinxupquote{ = 8}}}
\end{fulllineitems}

\index{TILT (SerialArmController.Command attribute)@\spxentry{TILT}\spxextra{SerialArmController.Command attribute}}

\begin{fulllineitems}
\phantomsection\label{\detokenize{src/serialarmcontroller:SerialArmController.Command.TILT}}\pysigline{\sphinxbfcode{\sphinxupquote{TILT}}\sphinxbfcode{\sphinxupquote{ = 9}}}
\end{fulllineitems}

\index{SHUTDOWN (SerialArmController.Command attribute)@\spxentry{SHUTDOWN}\spxextra{SerialArmController.Command attribute}}

\begin{fulllineitems}
\phantomsection\label{\detokenize{src/serialarmcontroller:SerialArmController.Command.SHUTDOWN}}\pysigline{\sphinxbfcode{\sphinxupquote{SHUTDOWN}}\sphinxbfcode{\sphinxupquote{ = 16}}}
\end{fulllineitems}


\end{fulllineitems}

\index{Position (class in SerialArmController)@\spxentry{Position}\spxextra{class in SerialArmController}}

\begin{fulllineitems}
\phantomsection\label{\detokenize{src/serialarmcontroller:SerialArmController.Position}}\pysiglinewithargsret{\sphinxbfcode{\sphinxupquote{class }}\sphinxcode{\sphinxupquote{SerialArmController.}}\sphinxbfcode{\sphinxupquote{Position}}}{\emph{\DUrole{n}{pitch}\DUrole{o}{=}\DUrole{default_value}{0}}, \emph{\DUrole{n}{yaw}\DUrole{o}{=}\DUrole{default_value}{0}}, \emph{\DUrole{n}{roll}\DUrole{o}{=}\DUrole{default_value}{0}}}{}
A class to store position data
\index{\_\_init\_\_() (SerialArmController.Position method)@\spxentry{\_\_init\_\_()}\spxextra{SerialArmController.Position method}}

\begin{fulllineitems}
\phantomsection\label{\detokenize{src/serialarmcontroller:SerialArmController.Position.__init__}}\pysiglinewithargsret{\sphinxbfcode{\sphinxupquote{\_\_init\_\_}}}{\emph{\DUrole{n}{pitch}\DUrole{o}{=}\DUrole{default_value}{0}}, \emph{\DUrole{n}{yaw}\DUrole{o}{=}\DUrole{default_value}{0}}, \emph{\DUrole{n}{roll}\DUrole{o}{=}\DUrole{default_value}{0}}}{}
Constructor for Position
\begin{quote}\begin{description}
\item[{Parameters}] \leavevmode\begin{itemize}
\item {} 
\sphinxstyleliteralstrong{\sphinxupquote{pitch}} \textendash{} The pitch value

\item {} 
\sphinxstyleliteralstrong{\sphinxupquote{yaw}} \textendash{} The yaw value

\item {} 
\sphinxstyleliteralstrong{\sphinxupquote{roll}} \textendash{} The roll value

\end{itemize}

\end{description}\end{quote}

\end{fulllineitems}


\end{fulllineitems}

\index{SerialArmController (class in SerialArmController)@\spxentry{SerialArmController}\spxextra{class in SerialArmController}}

\begin{fulllineitems}
\phantomsection\label{\detokenize{src/serialarmcontroller:SerialArmController.SerialArmController}}\pysiglinewithargsret{\sphinxbfcode{\sphinxupquote{class }}\sphinxcode{\sphinxupquote{SerialArmController.}}\sphinxbfcode{\sphinxupquote{SerialArmController}}}{\emph{\DUrole{n}{\_status\_bar}}, \emph{\DUrole{n}{\_notifications}}}{}
Send commands to the robot arm and receives data from the arm
\index{\_\_init\_\_() (SerialArmController.SerialArmController method)@\spxentry{\_\_init\_\_()}\spxextra{SerialArmController.SerialArmController method}}

\begin{fulllineitems}
\phantomsection\label{\detokenize{src/serialarmcontroller:SerialArmController.SerialArmController.__init__}}\pysiglinewithargsret{\sphinxbfcode{\sphinxupquote{\_\_init\_\_}}}{\emph{\DUrole{n}{\_status\_bar}}, \emph{\DUrole{n}{\_notifications}}}{}
Constructor for SerialArmController
\begin{quote}\begin{description}
\item[{Parameters}] \leavevmode\begin{itemize}
\item {} 
\sphinxstyleliteralstrong{\sphinxupquote{\_status\_bar}} \textendash{} StatusBar to use

\item {} 
\sphinxstyleliteralstrong{\sphinxupquote{\_notifications}} \textendash{} NotificationsFrame to use

\end{itemize}

\end{description}\end{quote}

\end{fulllineitems}

\index{update\_devs() (SerialArmController.SerialArmController method)@\spxentry{update\_devs()}\spxextra{SerialArmController.SerialArmController method}}

\begin{fulllineitems}
\phantomsection\label{\detokenize{src/serialarmcontroller:SerialArmController.SerialArmController.update_devs}}\pysiglinewithargsret{\sphinxbfcode{\sphinxupquote{update\_devs}}}{}{}
Updates the list of available devices

\end{fulllineitems}

\index{connect() (SerialArmController.SerialArmController method)@\spxentry{connect()}\spxextra{SerialArmController.SerialArmController method}}

\begin{fulllineitems}
\phantomsection\label{\detokenize{src/serialarmcontroller:SerialArmController.SerialArmController.connect}}\pysiglinewithargsret{\sphinxbfcode{\sphinxupquote{connect}}}{\emph{\DUrole{n}{comport}}}{}
Connects to the device at the desired COM port
\begin{quote}\begin{description}
\item[{Parameters}] \leavevmode
\sphinxstyleliteralstrong{\sphinxupquote{comport}} \textendash{} The name of the COM port to connect to

\end{description}\end{quote}

\end{fulllineitems}

\index{close() (SerialArmController.SerialArmController method)@\spxentry{close()}\spxextra{SerialArmController.SerialArmController method}}

\begin{fulllineitems}
\phantomsection\label{\detokenize{src/serialarmcontroller:SerialArmController.SerialArmController.close}}\pysiglinewithargsret{\sphinxbfcode{\sphinxupquote{close}}}{}{}
Closes the current connection

\end{fulllineitems}

\index{send() (SerialArmController.SerialArmController method)@\spxentry{send()}\spxextra{SerialArmController.SerialArmController method}}

\begin{fulllineitems}
\phantomsection\label{\detokenize{src/serialarmcontroller:SerialArmController.SerialArmController.send}}\pysiglinewithargsret{\sphinxbfcode{\sphinxupquote{send}}}{\emph{\DUrole{n}{data}}}{}
Sends data to the arm
\begin{quote}\begin{description}
\item[{Parameters}] \leavevmode
\sphinxstyleliteralstrong{\sphinxupquote{data}} \textendash{} Bytes to send

\end{description}\end{quote}

\end{fulllineitems}

\index{recv() (SerialArmController.SerialArmController method)@\spxentry{recv()}\spxextra{SerialArmController.SerialArmController method}}

\begin{fulllineitems}
\phantomsection\label{\detokenize{src/serialarmcontroller:SerialArmController.SerialArmController.recv}}\pysiglinewithargsret{\sphinxbfcode{\sphinxupquote{recv}}}{}{}
Receives data from the arm
\begin{quote}\begin{description}
\item[{Returns}] \leavevmode
data \sphinxhyphen{} bytes from the arm

\end{description}\end{quote}

\end{fulllineitems}

\index{update\_position() (SerialArmController.SerialArmController method)@\spxentry{update\_position()}\spxextra{SerialArmController.SerialArmController method}}

\begin{fulllineitems}
\phantomsection\label{\detokenize{src/serialarmcontroller:SerialArmController.SerialArmController.update_position}}\pysiglinewithargsret{\sphinxbfcode{\sphinxupquote{update\_position}}}{}{}
Updates the current stored position

\end{fulllineitems}

\index{set\_pitch() (SerialArmController.SerialArmController method)@\spxentry{set\_pitch()}\spxextra{SerialArmController.SerialArmController method}}

\begin{fulllineitems}
\phantomsection\label{\detokenize{src/serialarmcontroller:SerialArmController.SerialArmController.set_pitch}}\pysiglinewithargsret{\sphinxbfcode{\sphinxupquote{set\_pitch}}}{\emph{\DUrole{n}{val}}}{}
Sends a command to the arm to go to the desired pitch
\begin{quote}\begin{description}
\item[{Parameters}] \leavevmode
\sphinxstyleliteralstrong{\sphinxupquote{val}} \textendash{} The pitch to go to

\end{description}\end{quote}

\end{fulllineitems}

\index{set\_yaw() (SerialArmController.SerialArmController method)@\spxentry{set\_yaw()}\spxextra{SerialArmController.SerialArmController method}}

\begin{fulllineitems}
\phantomsection\label{\detokenize{src/serialarmcontroller:SerialArmController.SerialArmController.set_yaw}}\pysiglinewithargsret{\sphinxbfcode{\sphinxupquote{set\_yaw}}}{\emph{\DUrole{n}{val}}}{}
Sends a command to the arm to go to the desired yaw
\begin{quote}\begin{description}
\item[{Parameters}] \leavevmode
\sphinxstyleliteralstrong{\sphinxupquote{val}} \textendash{} The yaw to go to

\end{description}\end{quote}

\end{fulllineitems}

\index{set\_roll() (SerialArmController.SerialArmController method)@\spxentry{set\_roll()}\spxextra{SerialArmController.SerialArmController method}}

\begin{fulllineitems}
\phantomsection\label{\detokenize{src/serialarmcontroller:SerialArmController.SerialArmController.set_roll}}\pysiglinewithargsret{\sphinxbfcode{\sphinxupquote{set\_roll}}}{\emph{\DUrole{n}{val}}}{}
Sends a command to the arm to go to the desired roll
\begin{quote}\begin{description}
\item[{Parameters}] \leavevmode
\sphinxstyleliteralstrong{\sphinxupquote{val}} \textendash{} The roll to go to

\end{description}\end{quote}

\end{fulllineitems}

\index{close\_arm() (SerialArmController.SerialArmController method)@\spxentry{close\_arm()}\spxextra{SerialArmController.SerialArmController method}}

\begin{fulllineitems}
\phantomsection\label{\detokenize{src/serialarmcontroller:SerialArmController.SerialArmController.close_arm}}\pysiglinewithargsret{\sphinxbfcode{\sphinxupquote{close\_arm}}}{}{}
Sends the CLOSE command to the arm

\end{fulllineitems}

\index{open\_arm() (SerialArmController.SerialArmController method)@\spxentry{open\_arm()}\spxextra{SerialArmController.SerialArmController method}}

\begin{fulllineitems}
\phantomsection\label{\detokenize{src/serialarmcontroller:SerialArmController.SerialArmController.open_arm}}\pysiglinewithargsret{\sphinxbfcode{\sphinxupquote{open\_arm}}}{}{}
Sends the OPEN command to the arm

\end{fulllineitems}

\index{portrait() (SerialArmController.SerialArmController method)@\spxentry{portrait()}\spxextra{SerialArmController.SerialArmController method}}

\begin{fulllineitems}
\phantomsection\label{\detokenize{src/serialarmcontroller:SerialArmController.SerialArmController.portrait}}\pysiglinewithargsret{\sphinxbfcode{\sphinxupquote{portrait}}}{}{}
Sends the PORTRAIT command to the arm

\end{fulllineitems}

\index{landscape() (SerialArmController.SerialArmController method)@\spxentry{landscape()}\spxextra{SerialArmController.SerialArmController method}}

\begin{fulllineitems}
\phantomsection\label{\detokenize{src/serialarmcontroller:SerialArmController.SerialArmController.landscape}}\pysiglinewithargsret{\sphinxbfcode{\sphinxupquote{landscape}}}{}{}
Sends the LANDSCAPE command to the arm

\end{fulllineitems}

\index{tilt() (SerialArmController.SerialArmController method)@\spxentry{tilt()}\spxextra{SerialArmController.SerialArmController method}}

\begin{fulllineitems}
\phantomsection\label{\detokenize{src/serialarmcontroller:SerialArmController.SerialArmController.tilt}}\pysiglinewithargsret{\sphinxbfcode{\sphinxupquote{tilt}}}{}{}
Sends the TILT command to the arm

\end{fulllineitems}

\index{nod() (SerialArmController.SerialArmController method)@\spxentry{nod()}\spxextra{SerialArmController.SerialArmController method}}

\begin{fulllineitems}
\phantomsection\label{\detokenize{src/serialarmcontroller:SerialArmController.SerialArmController.nod}}\pysiglinewithargsret{\sphinxbfcode{\sphinxupquote{nod}}}{}{}
Sends the NOD command to the arm

\end{fulllineitems}

\index{shake() (SerialArmController.SerialArmController method)@\spxentry{shake()}\spxextra{SerialArmController.SerialArmController method}}

\begin{fulllineitems}
\phantomsection\label{\detokenize{src/serialarmcontroller:SerialArmController.SerialArmController.shake}}\pysiglinewithargsret{\sphinxbfcode{\sphinxupquote{shake}}}{}{}
Sends the SHAKE command to the arm

\end{fulllineitems}

\index{\_shutdown() (SerialArmController.SerialArmController method)@\spxentry{\_shutdown()}\spxextra{SerialArmController.SerialArmController method}}

\begin{fulllineitems}
\phantomsection\label{\detokenize{src/serialarmcontroller:SerialArmController.SerialArmController._shutdown}}\pysiglinewithargsret{\sphinxbfcode{\sphinxupquote{\_shutdown}}}{}{}
Sends the SHUTDOWN command to the arm

\end{fulllineitems}


\end{fulllineitems}



\chapter{StatusBar}
\label{\detokenize{src/statusbar:module-StatusBar}}\label{\detokenize{src/statusbar:statusbar}}\label{\detokenize{src/statusbar::doc}}\index{module@\spxentry{module}!StatusBar@\spxentry{StatusBar}}\index{StatusBar@\spxentry{StatusBar}!module@\spxentry{module}}
Created on Thu Feb 27 14:17:02 2020

@author: shill
\index{StatusBar (class in StatusBar)@\spxentry{StatusBar}\spxextra{class in StatusBar}}

\begin{fulllineitems}
\phantomsection\label{\detokenize{src/statusbar:StatusBar.StatusBar}}\pysiglinewithargsret{\sphinxbfcode{\sphinxupquote{class }}\sphinxcode{\sphinxupquote{StatusBar.}}\sphinxbfcode{\sphinxupquote{StatusBar}}}{\emph{\DUrole{n}{master}}, \emph{\DUrole{o}{**}\DUrole{n}{kwargs}}}{}
Displays the connection status of the GUI to the arm
\index{\_\_init\_\_() (StatusBar.StatusBar method)@\spxentry{\_\_init\_\_()}\spxextra{StatusBar.StatusBar method}}

\begin{fulllineitems}
\phantomsection\label{\detokenize{src/statusbar:StatusBar.StatusBar.__init__}}\pysiglinewithargsret{\sphinxbfcode{\sphinxupquote{\_\_init\_\_}}}{\emph{\DUrole{n}{master}}, \emph{\DUrole{o}{**}\DUrole{n}{kwargs}}}{}
Constructor for StatusBar
\begin{quote}\begin{description}
\item[{Parameters}] \leavevmode
\sphinxstyleliteralstrong{\sphinxupquote{master}} \textendash{} The Tk parent widget

\end{description}\end{quote}

\end{fulllineitems}

\index{set\_status() (StatusBar.StatusBar method)@\spxentry{set\_status()}\spxextra{StatusBar.StatusBar method}}

\begin{fulllineitems}
\phantomsection\label{\detokenize{src/statusbar:StatusBar.StatusBar.set_status}}\pysiglinewithargsret{\sphinxbfcode{\sphinxupquote{set\_status}}}{\emph{\DUrole{n}{status}}}{}
Sets the status of the GUI to the arm
\begin{quote}\begin{description}
\item[{Parameters}] \leavevmode
\sphinxstyleliteralstrong{\sphinxupquote{status}} \textendash{} The status to set to

\end{description}\end{quote}

\end{fulllineitems}


\end{fulllineitems}



\renewcommand{\indexname}{Python Module Index}
\begin{sphinxtheindex}
\let\bigletter\sphinxstyleindexlettergroup
\bigletter{a}
\item\relax\sphinxstyleindexentry{Application}\sphinxstyleindexpageref{src/application:\detokenize{module-Application}}
\indexspace
\bigletter{c}
\item\relax\sphinxstyleindexentry{ControlButtons}\sphinxstyleindexpageref{src/controlbuttons:\detokenize{module-ControlButtons}}
\indexspace
\bigletter{n}
\item\relax\sphinxstyleindexentry{NotificationsFrame}\sphinxstyleindexpageref{src/notificationsframe:\detokenize{module-NotificationsFrame}}
\indexspace
\bigletter{p}
\item\relax\sphinxstyleindexentry{PositionFrame}\sphinxstyleindexpageref{src/positionframe:\detokenize{module-PositionFrame}}
\indexspace
\bigletter{s}
\item\relax\sphinxstyleindexentry{SerialArmController}\sphinxstyleindexpageref{src/serialarmcontroller:\detokenize{module-SerialArmController}}
\item\relax\sphinxstyleindexentry{StatusBar}\sphinxstyleindexpageref{src/statusbar:\detokenize{module-StatusBar}}
\end{sphinxtheindex}

\renewcommand{\indexname}{Index}
\printindex
\end{document}